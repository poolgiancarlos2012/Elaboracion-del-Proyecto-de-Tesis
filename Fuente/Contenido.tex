\newpage

\section{PROBLEMA DE INVESTIGACIÓN}
\subsection{Planteamiento del Problema}

%\blindtext % crea un trozo de texto corto
% \Blindtext % crea un trozo de texto largo
% \lipsum[5-6] % Párrafos 5 y 6 creados con Lorem Ipsum

% Este es un ejemplo de un grupo de parrafo
% \large 
% \doublespacing
% \lipsum[17]
% \singlespacing

\large 
\doublespacing

%PLANTEAMIENTO DEL PROBLEMA RELACIONADO CON V.I.
Actualmente EEUU y China están a la vanguardia de la modernidad por ser los mayores productores de tecnología, las webs líderes en comercio electrónico vienen de estos países por lo tanto ya es una necesidad que las empresas cuente con una página web, que siempre están disponibles sin importar el lugar y el momento, ofreciendo las 24 horas sus productos y servicios, transmitiendo el conocimiento exacto de lo que cliente busca por otro lado el mercado global de teléfonos inteligentes seguirá en crecimiento ya que los positivos aportes de la tecnología móvil como el rastreo, autentificación de los compradores así como smartphones con mejor capacidad o aplicaciones de video como Netflix o HBO crean necesidades de seguir adquiriendo estos dispositivos por ende los nuevos consumidores le dedican cada vez más tiempo a estos.

América Latina y el Caribe ha tenido un vertiginoso avance en dar acceso a internet en estos en estos últimos años, actualmente el 44\% de ALC dispone de internet que es casi la mitad de la población, a pesar de esto aún queda realizar cambios en la gobernanza de las telecomunicaciones con el fin de modernizar la infraestructura de comunicaciones y reducir la brecha digital con las otras regiones y preparar a los países a una economía más digital. Así mismo las redes Móviles son las que más crecimiento han tenido a diferencia de los que tienen acceso a internet, esto debido a que este crecimiento es impulsado por el consumo de video en línea, redes sociales, streaming entre otros, el futuro de éxito dependerá de las políticas flexibles que incentive la inversión de los operadores en redes y permita a los usuarios acceder a los beneficios de una conectividad móvil de alta calidad

La interacción tecnológica en línea en el Perú aún está en crecimiento, en la capital casi ya se masifico el acceso a internet por medio de una computadora, Celular o Smartphone. Pero aún falta avanzar en dar acceso a provincias o zonas alejadas. Por lo general los usuarios que realizan operaciones complejas en internet son jóvenes o mayores de hasta 34 años, es decir realizan Reservas, compras, trasferencia o Pagos por internet. En cuanto uso de Celular o Smartphone prácticamente se masifico en todo el país, pero aun así la velocidad de internet móvil no se ajusta al mercado, es por eso que en la actualmente la cobertura 4g casi alcanza la mitad del país. Perú tiene que ver la forma de cómo superar los retos en el futuro, tomando en cuenta la nueva cobertura 5G que se avecinara en 2 o 3 años, para esta nueva tecnología se necesitara seguir desarrollando infraestructura de telecomunicaciones para dar a todo el país acceso a internet.

%CITA REFERENCIAL V.I.
Para Cabero las TIC: “En líneas generales podríamos decir que las nuevas tecnologías de la información y comunicación son las que giran en torno a tres medios básicos: la informática, la microelectrónica y las telecomunicaciones; pero giran, no sólo de forma aislada, sino lo que es más significativo de manera interactiva e interconexionadas, lo que permite conseguir nuevas realidades comunicativas”. (Cabero, 1998: 198) 

%PLANTEAMIENTO DEL PROBLEMA RELACIONADO CON V.D.
Estados Unidos tiene complejos canales de distribución lo cual permite el desarrollo de su mercado mostrándose más competitivo y con mayor demanda pues es importante conocer la gran diversidad de clientes, precios de bienes y servicios, cultura del país y los competidores así mismo las diferentes soluciones de pago o transferencia permitieron a la mayoría de los consumidores optar por una compra rápida y segura, por lo tanto este es un mercado menos tradicional la cual valora mucho la información, tiempo y confianza.

En América Latina si bien hay varias características similares entre los países de esta región también hay diferencias culturales, la diferencia en términos de competitividad en ventas y Marketing internacional con Latinoamérica es mucho, pero aun así está región está en una etapa de transición pues aún le falta superar algunas barreras como la entrega y logística, métodos de pago y temas legales, pero se reconoce que crece año tras año.

La influencia internacional exige al Perú a alinearse de manera gradual en sus diferentes sectores productivos. Ante los nuevos métodos de compra o pago los peruanos son desconfiados ya que es difícil cambiarles de costumbre, en la mayoría de casos en esta región prevalece las ventas tradicionales por ello los peruanos valoran que las ventas se desarrolle en un entorno amigable, directo y simple, a pesar de esto el Perú poco a poco está migrando a nuevas herramientas digitales que le permitan abrir nuevas oportunidades en el comercio nacional e internacional y en un futuro cercano el  mercado interno sea más dinámico y competitivo.

Actualmente en Lima por lo general las ventas se desarrolla de manera tradicional, la mayor proporción de la población de esta región expresa mucha desconfianza o desconocimiento de estas nuevas alternativas de compra y otro pequeño porcentaje está cambiando esta perspectiva  por una compra más rápida que les ahorre tiempo, esta región del país está encaminándose a desarrollar nuevas estrategias para vender, los procesos de compra y medios de pago deben mejorar por una más amigable, segura y confiable, en unos años este cambio se notara aún más.

El rubro de la empresa Grupo Andina S.A.C. es el mercado agrícola, brinda soluciones a través de productos agrícolas, que adicionen valor a los cultivos de sus clientes. Por lo tanto, los procesos bases de esta empresa son el área de Comercial, Compras y Almacenamiento.

Inicialmente la empresa empezó en las provincias cercanas a Lima como Cañete, Ica, Pisco etc. En esos tiempos solo se contaba con teléfono fijo y mal servicio de llamadas a larga distancia luego se fueron expandiendo gradualmente a nivel nacional, cabe destacar que con el dispositivo que cuentan disponible en su labor actualmente es con un móvil.

Este proyecto se centrara en el área de Ventas ya que se encontró que unos de los principales problemas es que hay un bajo control de las actividades en las fuerza de ventas de la empresa, esto se debe por la poca automatización de los procesos, es decir semanalmente los vendedores emiten alrededor de 40 formatos que informan sobre sus actividades que por lo general los supervisores asignados no lo leen por la gran cantidad de información, en cada formato se incluye información de los cliente, cultivos, productos, ubicaciones entre otros detalles que en síntesis contiene un conjunto de información valiosa, además los supervisores tienen que salir al campo a visitar a los clientes además de apoyar con la cobranza y venta. Todo esto provoca que el supervisor no pueda ver si los vendedores están cumpliendo una eficiente labor, además que los índices en ventas están muy bajos, no se puede controlar las rutas o recorridos de los vendedores, que incurren a altos gastos en combustible los cuales no justifica el nivel de ventas obtenido. Por la falta de información hay pérdida de oportunidades de negocios con los clientes nuevos. También genera que se pierda clientes regulares por falta de atención o visita, baja rentabilidad por los excesivos gastos incurridos, perdida en ventas y desaprovechar otros mercados por no contar con información oportuna.

% CITA REFERENCIAL V.D.
Según Michael Porter en Ventaja competitiva: creación y sostenimiento de un desempeño superior, “la esencia de implantar estrategias competitivas radica en establecer la relación existente entre la empresa y su entorno. Los integrantes de la fuerza de ventas son los que disponen de este contacto directo con el micro y macro ambiente y con el entorno competitivo tanto interno, como externo de la empresa”.

\singlespacing

\newpage

\subsection{Formulación del problema}
\subsubsection{Problema General}

\begin{itemize}[leftmargin=1cm]
\item Bajo nivel de control de las actividades de la fuerza de ventas de la empresa
\end{itemize}

\subsubsection{Problemas Especificos}

%\doublespacing

\begin{itemize}[leftmargin=1cm]
\item Bajo conocimiento de la efectividad de los vendedores.
\item Escaso control de las rutas o recorridos de los vendedores.
\item Pérdida de oportunidades de negocio con los clientes visitados.
\item Pérdida de clientes regulares por falta de atención.
\item Perdida en las ventas.
\item Desaprovechar oportunidades de explotar otros mercados.
\item Baja rentabilidad, por lo excesivos gastos incurridos.
\end{itemize} 



%\singlespacing
%\blindtext

\subsection{Justificación y aportes}

\large 
\doublespacing

La presente investigación se seleccionó debido a la necesidad actual de poder solucionar los problemas de la empresa como los gastos excesivos por la falta de control a los vendedores que laboran en el campo y las pérdidas económicas por la poca automatización de los procesos, a si mismo para la obtención del título profesional ya que por el cargo actual que me que me desempeño me facultan  poder aportar con esta propuesta de solución a la problemática actual.

\singlespacing

%\blindtext

\subsection{Objetivos de la Investigación}

\subsubsection{Objetivo General}

\begin{itemize}[leftmargin=1cm]
\item Controlar las actividades en las fuerza de ventas de la empresa.
\end{itemize}

\subsubsection{Objetivos Especificos}

%\doublespacing

\begin{itemize}[leftmargin=1cm]
\item Conocer la efectividad de los vendedores
\item Controlar las rutas o recorridos de los vendedores
\item Aprovechar oportunidades de negocio con los clientes visitados
\item Mantener contacto con los clientes regulares
\item Incrementar las ventas
\item Aprovechar oportunidades de explotar otros mercados
\item Aumentar rentabilidad de la empresa

\end{itemize} 

%\blindtext



\section{MARCO METODOLÓGICO}
\subsection{Antecedentes de la investigación}
\subsection{Bases teóricas de las variables}
\subsection{Definición de términos básicos}

\section{METODOS Y MATERIALES}
\subsection{Hipótesis de la investigación}
\subsection{Variables}
\subsection{Operacionalizacion de la variable}
\subsection{Diseño de la investigación}
\subsection{Población, muestra y muestreo}
\subsection{Técnicas e instrumentos de recolección de datos}
\subsection{Métodos de análisis de datos}
\subsection{Propuesta de valor}
\subsection{Aspectos deontológicos}

\section{ASPECTOS ADMINISTRATIVOS}
\subsection{Presupuesto}
\subsection{Cronograma de actividades}

\section{REFERENCIAS}
