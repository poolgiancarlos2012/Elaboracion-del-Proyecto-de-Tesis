\newpage

\section{PROBLEMA DE INVESTIGACIÓN}
\subsection{Planteamiento del Problema}

%\blindtext % crea un trozo de texto corto
% \Blindtext % crea un trozo de texto largo
% \lipsum[5-6] % Párrafos 5 y 6 creados con Lorem Ipsum

% Este es un ejemplo de un grupo de parrafo
% \large 
% \doublespacing
% \lipsum[17]
% \singlespacing

\large 
\doublespacing
Frente a intensa competencia en las ventas, las empresas deben ver la forma de mantenerse en el mercado por eso es necesario sacar el máximo provecho de todos los recursos relacionados a la venta. En opinión de Philip Kotler y Gary Armstrong en Fundamentos del Marketing, "la fuerza de ventas representa, por un lado, a la empresa ante los consumidores, facilitándoles información, asesoramiento y ayuda. Además, actúa como portavoz de los clientes ante la organización, trasladando a la Dirección sus quejas, sugerencias o preocupaciones.". Según Michael Porter en Ventaja competitiva: creación y sostenimiento de un desempeño superior, “la esencia de implantar estrategias competitivas radica en establecer la relación existente entre la empresa y su entorno. Los integrantes de la fuerza de ventas son los que disponen de este contacto directo con el micro y macro ambiente y con el entorno competitivo tanto interno, como externo de la empresa”.

%\hfill

%\lipsum[5-6]

%\vspace{5mm}
%\linebreak
%\vfill

Estados Unidos tiene complejos canales de distribución lo cual permite el desarrollo de su mercado mostrándose más competitivo y con mayor demanda pues es importante conocer la gran diversidad de clientes, precios de bienes y servicios, cultura del país y los competidores así mismo las diferentes soluciones de pago o transferencia permitieron a la mayoría de los consumidores optar por una compra rápida y segura, por lo tanto este es un mercado menos tradicional la cual valora mucho la información, tiempo y confianza.



%\hfill 
%\vfill

En América Latina si bien hay varias características similares entre los países de esta región también hay diferencias culturales, la diferencia en términos de competitividad en ventas y Marketing internacional con Latinoamérica es mucho, pero aun así está región está en una etapa de transición pues aún le falta superar algunas barreras como la  entrega y logística, métodos de pago y temas legales pero se reconoce que crece año tras año.

%\newline

%\hfill 
%\vfill
La influencia internacional exige al Perú a alinearse de manera gradual en sus diferentes sectores productivos. Ante los nuevos métodos de compra o pago los peruanos son desconfiados ya que es difícil cambiarles de costumbre, en la mayoría de casos en esta región prevalece las ventas tradicionales por ello los peruanos valoran que las ventas se desarrolle en un entorno amigable, directo y simple, a pesar de esto el Perú poco a poco está migrando a nuevas herramientas digitales que le permitan abrir nuevas oportunidades en el comercio nacional e internacional y en un futuro cercano el  mercado interno sea más dinámico y competitivo.

Actualmente en Lima por lo general las ventas se desarrolla de manera tradicional, la mayor proporción de la población de esta región expresa mucha desconfianza o desconocimiento de estas nuevas alternativas de compra y otro pequeño porcentaje está cambiando esta perspectiva  por una compra más rápida que les ahorre tiempo, esta región del país está encaminándose a desarrollar nuevas estrategias para vender, los procesos de compra y medios de pago deben mejorar por una más amigable, segura y confiable, en unos años este cambio se notara aún más.

Actualmente EEUU y China están a la vanguardia de la modernidad por ser los mayores productores de tecnología, las webs líderes en comercio electrónico vienen de estos países por lo tanto ya es una necesidad que las empresas cuente con una página web, que siempre están disponibles sin importar el lugar y el momento, ofreciendo las 24 horas sus productos y servicios, transmitiendo el conocimiento exacto de lo que cliente busca por otro lado el mercado global de teléfonos inteligentes seguirá en crecimiento ya que los positivos aportes de la tecnología móvil como el rastreo, autentificación de los compradores así como smartphones con mejor capacidad o aplicaciones de video como Netflix o HBO crean necesidades de seguir adquiriendo estos dispositivos por ende los nuevos consumidores le dedican cada vez más tiempo a estos.


America Latina y el Caribe a tenido un vertiginoso avance en dar acceso a internet en estos en estos ultimos años, actualmente el 44\% de ALC dispone de internet que es casi la mitad de la poblacion, a pesar de esto aun queda realizar cambios en la gobernanza de las telecomunicaciones con el fin de modernizar la infraestrucutra de comunicaciones y reducir la brecha digital con las otras regiones y preparar a los paises a una economia mas digital. Asi mismo las redes Moviles son las que mas crecimiento han tenido a diferencia de los que tienen acceso a internet, esto debido a que este crecimiento es impulsado por el consumo de video en linea, redes sociales, streaming entre otros, el futuro de exito dependera de las políticas flexibles que incentive la inversión de los operadores en redes y permita a los usuarios acceder a los beneficios de una conectividad móvil de alta calidad

La interacción tecnológica en línea en el Perú aún está en crecimiento, en la capital casi ya se masifico el acceso a internet por medio de una computadora, Celular o Smartphone. Pero aún falta avanzar en dar acceso a provincias o zonas alejadas. Por lo general los usuarios que realizan operaciones complejas en internet son jóvenes o mayores de hasta 34 años, es decir realizan Reservas, compras, trasferencia o Pagos por internet. En cuanto uso de Celular o Smartphone prácticamente se masifico en todo el país, pero aun así la velocidad de internet móvil no se ajusta al mercado, es por eso que en la actualmente la cobertura 4g casi alcanza la mitad del país. Perú tiene que ver la forma de cómo superar los retos en el futuro, tomando en cuenta la nueva cobertura 5G que se avecinara en 2 o 3 años, para esta nueva tecnología se necesitara seguir desarrollando infraestructura de telecomunicaciones para dar a todo el país acceso a internet.

\singlespacing

\subsection{Formulación del problema}

%\blindtext

\subsection{Justificación y aportes}

%\blindtext

\subsection{Objetivos de la Investigación}

%\blindtext



\section{MARCO METODOLÓGICO}
\subsection{Antecedentes de la investigación}
\subsection{Bases teóricas de las variables}
\subsection{Definición de términos básicos}

\section{METODOS Y MATERIALES}
\subsection{Hipótesis de la investigación}
\subsection{Variables}
\subsection{Operacionalizacion de la variable}
\subsection{Diseño de la investigación}
\subsection{Población, muestra y muestreo}
\subsection{Técnicas e instrumentos de recolección de datos}
\subsection{Métodos de análisis de datos}
\subsection{Propuesta de valor}
\subsection{Aspectos deontológicos}

\section{ASPECTOS ADMINISTRATIVOS}
\subsection{Presupuesto}
\subsection{Cronograma de actividades}

\section{REFERENCIAS}
